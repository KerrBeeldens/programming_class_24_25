\chapter{Day 2: flow \& conditionals}

\section{Summary of topics}
% Summarize the topics from the “listen” segment of the day (150 – 250 words)
The first topic of the lesson of today was about flow, which is the order in which the lines are executed. Yesterday's programs where executed from the top line to the bottom line, but this does not have to be the case. This can be achieved using the \texttt{setup()} and \texttt{draw()} methods. The \texttt{setup()} method is only ran once when the program launches. After that, the \texttt{draw()} method is repeatedly executed.

The second topic was about conditionals. A conditional takes the following form:

\begin{codebox}{example 2.1}
    \begin{lstlisting}
if (relational expression) {
    // lines to be executed if true
}
    \end{lstlisting}
\end{codebox}

In here the relational expression is a statement which can be \texttt{true} or \texttt{false}. If the statement is true, the lines within the conditional are executed. Else, they are ignored. A relational statement takes the following form:

\begin{codebox}{example 2.2}
    \begin{lstlisting}
7 > 12 // > means greater than, so this is false
7 < 12 // > means less than, so this is true
7 == 12 // == means equal to, so this is false
7 != 12 // != means not equal to, so this is true
    \end{lstlisting}
\end{codebox}

One can also combine add the \texttt{=} symbol to the less than or greater than symbols to mean less than or equal to/greater than or equal to. Next, it was also explained that with the \texttt{else} statement, you can tell the program what to do if a conditional is false:

\begin{codebox}{example 2.3}
    \begin{lstlisting}
if (relational expression) {
    // lines to be executed if true
} else {
    // lines to be executed if false
}
    \end{lstlisting}
\end{codebox}

Two unrelated topics that where briefly mentioned today where those of indentation and the mouse location. Indentation is the convention to offset lines within blocks with a tab to the right. This makes your program easier to read. The location of the mouse is stored in the variables \texttt{mouseX} and \texttt{mouseY}.


\section{Challenge description: \textcolor{red}{name}}
% A description of that day’s challenge describing what the assignment was, what you tried to achieve and how you applied the topics from the “listen” segment. include instruction on how to use is and Include screenshots/screen captures. (150 – 250 words)
